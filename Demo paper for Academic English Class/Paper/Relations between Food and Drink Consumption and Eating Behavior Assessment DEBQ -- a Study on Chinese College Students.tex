%File: formatting-instruction.tex
\documentclass[letterpaper]{article}
\usepackage{style}
\usepackage{times}
\usepackage{helvet}
\usepackage{courier}

\usepackage{booktabs}
\usepackage{verbatim}
\usepackage{enumerate}
\usepackage{graphicx}

\frenchspacing
\setlength{\pdfpagewidth}{8.5in}
\setlength{\pdfpageheight}{11in}
\pdfinfo{
/Title (Relations between Food and Drink Consumption and Eating Behavior Assessment DEBQ -- a Study on Chinese College Students)
/Author (Jing Lan)}
\setcounter{secnumdepth}{0}  
\begin{document}

\title{Relations between Food and Drink Consumption and Eating Behavior Assessment DEBQ -- a Study on Chinese College Students}
\author{Jing Lan\\
Sun Yat-sen University\\
No. 135, Xingang Xi Road \\
Guangzhou, 510275, P. R. China \\
}

\maketitle

\begin{abstract}
\begin{quote}
    Eating disorders and eating behavior patterns have been studied for decades.
    Study around the late 20th century focused majorly on generalizing clinical patterns 
    of eating disorders for diagnostic purposes. However, it was in recent years that researchers
    have found prevalence of abnormal eating behaviors, emotional eating for instance, from people who don't necessarily meet clinical
    disorder criteria well. Furthermore, study shown that no exclusive measurement of general eating behaviors is available so far. However, 
    few self-administered questionnaires on reporting eating disorders and corresponding patterns, for example, EAT-40, EDI and DEBQ, have shown
    considerable effectiveness on differentiating general abnormal eating patterns on non-clinical samples. In order to replicate robustness of DEBQ, one of the most
    widely-used eating questionnaires, on Chinese college student samples as well as predicting their food/drink choices and intake, measured by SDSCA-style scales,
    we collected 52 pieces of DEBQ from SYSU students. Some relations between three DEBQ subscales and food/drink consumption have been established.
    Also, implementation of machine learning algorithms provides promising potential for 
    relation discovery and symptom prediction for future study.  
\end{quote}
\end{abstract}

\subsection{Keywords}
DEBQ, food/drink consumption, college students, machine learning

\section{Objective}
\subsection{Background}
Prior study on abnormal eating behavior patterns and eating disorders began with
characterizing distinct kinds of eating disorders, inventing accurate diagnostic
criteria as well as effective measurement of the extent of abnormal symptoms. The Eating Attitude Test(EAT), 
consisting of 40 items(EAT-40), was the first self-reported questionnaire for evaluation of
eating disorder patterns\cite{garner1979eating}. Later, a more efficient version EAT-26 was introduced
still with great performance in validation\cite{garner1982eating}. \\
\indent However, as research went further, the reliability of EAT was challenged. 
First, clinical criteria of anorexia nervosa have since changed greatly along with the development of pathology,
meaning the original clinical standard\cite{feighner1972diagnostic} used for EAT validation could be obsolete.
Also, more types of eating disorders have been separated into different categories. For instance, \cite{russell1979bulimia} first separated 
bulimia nervosa(overeating) from anorexia nervosa, leading to the distinction between the two most common eating disorders that are indistinguishable for 
the original EAT-40. \\
\indent Consequently, the original EAT test was then validated and used as a general but still effective assessment for abnormal 
eating patterns instead of specific clinical disorders\cite{carter1984screening,gross1986validity}. This was when abnormal eating patterns 
were discovered from non-clinical samples(at least under "modern" criteria). More questionnaires with the same aim
were invented, including the Eating Disorder Inventory(EDI)\cite{garner1983development}, the Dutch Eating Behavior Questionnaire(DEBQ)\cite{van1986dutch}, 
the Emotional Eating Questionnaire(EMAQ)\cite{geliebter2003emotional} and so forth. Usually higher scores in the tests indicate worse eating patterns\\
\indent Questionnaires on detection of bad eating patterns, such as DEBQ and EMAQ, usually derived from few but major theories on the formative
mechanism inside eating disorders. For instance, the psychosomatic theory\cite{bruch1964psychological} corresponds to EMAQ and the emotional part of DEBQ.
Amid several validated and reliable questionnaires, DEBQ presents high theoretical flexibility while 
maintaining extraordinary reliability\cite{hyland1989psychometric,braet1997assessment,nolan2010emotional},
which made it an effective tool for detection of abnormal eating patterns. \\
\indent For DEBQ, few questions are still under-researched. Firstly, until recent years, it wasn't systematically validated on Chinese samples even translated into Chinese by researchers\cite{wu2017validation,wang2018psychometric}.
Although researchers supported its effectiveness on some Chinese samples, there are reasons to worry about. 
For example, questions in the \textbf{External Eating} section like \textbf{"If you walk pass the baker do you have the desire to buy something delicious?"} may not work well for 
Chinese samples considering an eating environment with much fewer \textbf{bakeries} and \textbf{cafes} and totally different eating habits. 
Plus, the self-report biases can significantly reduce the effectiveness of the questionnaire.
Consequently, the reliability and effectiveness of DEBQ were tested in this research.
Secondly, positive relation has been found in prior study between restrained eating and emotional eating patterns\cite{van1986predictive,wardle1987eating,macht2002chocolate}, 
meaning that restrained eating can bring adverse consequences to one's eating health, which is especially consistent with common sense today.
Given the prevalence of unhealthy eating patterns among young adults\cite{neumark2011dieting}, we also studied
this relation in the research. 

\subsection{Objectives}
In this research, we were to acheive the following objectives.
\begin{enumerate}
    \item to replicate reliability of DEBQ on a sample set of Chinese college students.
    \item to discover relations between three DEBQ subscales, gender, and food/drink consumption, assessed by SDSCA-type scales.
    \item as a minor objective, positive relation between restrained eating and unhealthy eating patterns 
          is expected. Gender differences will also be tested. 
\end{enumerate}

\section{Method}
\subsection{Participants} 
Participants of this study were 52 college students from Sun Yat-sen University, a public research university 
in China. Demographic information is presented in Table 1. All 31 males and 21 females
were young Asian adults, with average BMI \(20.8~(\pm 2.67)\) for male samples and \(19.0~(\pm 1.82)\) for females.
Demographic analysis shows typicality of samples used for the research.
Few samples were discarded in demographic analysis as participants responded with abnormal values due to privacy concerns.

\renewcommand\tablename{\textbf{Table}}
\begin{table}[htbp]
    \caption{Sample demographics (\textit{N}=52)}
    \centering
    \begin{tabular}{lcr} 
        \toprule
        Characteristics & average (\textit{N}) & precentage(range) \\
        \midrule
        Female (\textit{N}=21) &  & 0.404 \\ \\
        Age (\textit{N}=20) & \(19.25~(\pm 0.44)\) & (19-20)\\ \\
        BMI (\textit{N}=20) & \(19.1~(\pm 1.82)\) & (16.7-23.7) \\
        ~~~Underweight & (9) & 0.45 \\
        ~~~Normal Weight & (11) & 0.55 \\
        ~~~Overweight & (0) & 0 \\
        ~~~Obese & (0) & 0 \\
        \midrule
        Male (\textit{N}=31) &  & 0.596 \\ \\
        Age (\textit{N}=28) & \(19.35~(\pm 0.56)\) & (19-20)\\ \\
        BMI (\textit{N}=30) & \(20.8~(\pm 2.67)\) & (14.9-24.1)\\ 
        ~~~Underweight & (3) & 0.1 \\
        ~~~Normal Weight & (24) & 0.8 \\
        ~~~Overweight & (3) & 0.1 \\
        ~~~Obese & (0) & 0 \\
        \bottomrule
    \end{tabular}
\end{table}

\subsection{Measures}
\subsubsection{Demographic information}
Participants were asked to answer several questions about their
age, gender, stature and weight information for demographic analysis. BMI
 was not directly asked but computed using stature and weight data.
Validity of self-reported collection has been supported by prior research\cite{kuczmarski2001effects}.
We introduced Chinese instead of the WHO BMI categorization standard for grouping BMI information from
a sample set of predominantly Chinese students. The result was introduced as control variables to
ensure a typical sample set. Furthermore, we included gender as a variable in statistical analysis for 
discovering sexual differences in food/drink intake.

\subsubsection{Food and Drink Consumption}
Participants answered four SDSCA-style questions about their consumption
frequency on food and unhealthy drink. The Summary of Diabetes Self-Care
Activities(SDSCA), was introduced aiming at measuring diabetes self-management \cite{toobert2000summary}.
Part \textbf{Diet} of the self-administered questionnaire includes questions that well measure fatty food and vegetable consumption frequency.
We adapted the question form and reduced day range from 7 days to 4 for simplification and higher reliability, considering self-report biases.
Each question asked \textbf{"How many of the last four days did you eat(drink) ..."}, in order to measure participants' 
tendency of healthy/unhealthy intake. Concretely, four questions have been asked as follows:
\begin{table}[htbp]
    \caption{Questions on Food/Drink Consumption}
    \centering
    \begin{tabular}{l}
        \toprule
        Questions (range:0-4) \\
        \midrule
        How many of the last four days did you: \\
        1. have snacks or late supper besides the normal meals? \\
        2. eat four or more servings of vegetables and fruits? \\
        3. eat any high-fat food or full-fat dairy product? \\
        4. drink high-sugar soft drink? \\
        \bottomrule
    \end{tabular}
\end{table}

\subsubsection{Eating Behavior Patterns}
Participants were then asked to answer 25 Likert-style DEBQ questions evaluating
their eating behavior patterns. DEBQ is a widely-used, self-reported questionnaire with
three subscales\cite{van1986dutch}, each derived from one of the three main formative theories of eating disorders, 
and has since been validated as with high reliability and validity
\cite{van1986predictive,hyland1989psychometric,wardle1987eating,braet1997assessment}.
Effectiveness of DEBQ as well as its Chinese version C-DEBQ has also been verified recently 
\cite{wu2017validation,wang2018psychometric}. In the study, 7 questions on \textbf{External Eating},
9 questions on \textbf{Emotional Eating} and 9 questions on \textbf{Restrained Eating}
were translated into Chinese and randomly placed in the electronic question sheet in order to reduce self-report biases.
Three External Eating questions were removed due to their low critical ratios (CR\( < 3.00\))
\cite{wu2017validation}. We also removed four questions on diffused emotional
eating that were not relevant in order to simplify the questionnaire.

\subsection{Procedure}
The Web-based questionnaire was distributed through online communication platforms.
Participants who volunteered to take part in used their cellphones to answer the questionnaires.
They sequentially answered demographic questions, food/drink consumption questions and finally
DEBQ eating behavior questions. Results were then submitted and immediately collected by the Web-based questionnaire platform.
All 52 participants returned fully answered, valid questionnaires for further analysis. 
No specific personal data was asked or collected throughout the research procedure. 

\subsection{Statistical Analysis}
After all questionnaires and corresponding data have been collected by the online questionnaire platform,
We first tested the reliability of the questionnaire through \textbf{SPSS} statistical analysis for Cronbach's \(\alpha\) indexes
on each part of the questionnaire separately then as a whole. \\
\indent Then, \textbf{MATLAB} program script was launched to
rearrange all data, filter extreme, abnormal values in demographic information,
then produce final demographic statistics and three subscales in DEBQ parts. \\
\indent Finally, we took advantage of the third-party Python package \textbf{sklearn}\cite{scikit-learn} to implement
concise and efficient machine learning algorithms to reveal relations between variables.
\textbf{Sklearn} is an integrated machine learning package that implements several well-designed 
software tools that support data processing and accessible machine learning algorithms\cite{sklearn_api}.
In this research, gender and three subscales of DEBQ were treated as input, while food/drink 
consumption indexes were tested as output.

\section{Results}
\subsection{Reliability}
We first tested the reliability of DEBQ. Cronbach's \(\alpha\) indexes have been computed for 
each part of the questionnaire and as a whole. \textbf{Table 3} shows the results.
\begin{table}[htbp]
    \caption{Reliability measured by \(\alpha\)}
    \centering
    \begin{tabular}{lr}
        \toprule
        part & reliability \\
        \midrule
        DEBQ & 0.921 \\
        ~~Emotional Eating & 0.964 \\
        ~~Restrained Eating & 0.910 \\
        ~~External Eating & 0.587 \\
        \bottomrule
    \end{tabular}
\end{table}
The overall reliability of DEBQ is solid(\(>= 0.9\)). Also, \textbf{Emotional Eating} and
\textbf{Restrained Eating} parts present perfect reliability. However, the \textbf{External Eating} part present
relatively poor reliability, which is consistent with our hypothesis and prior study.

\subsection{Relation Discovery}
\textbf{Figure 1, 2 and 3} present the diagrams indicating relations
between four SDSCA-type results and three DEBQ eating subscales. Due to the limited 
size of the data set, the overall relations between variables are not prominently clear.
Male samples are labeled yellow in the diagrams, while females are labeled red.
As for the curves, yellow and red lines corresponds to male and female samples, respectively.
Those blue lines are for the whole data set.
\begin{figure}[h]
    \caption{\textbf{Emotional Eating}}
    \centering
    \includegraphics[scale=0.4]{emo.png}
\end{figure}
\begin{figure}[h]
    \caption{\textbf{External Eating}}
    \centering
    \includegraphics[scale=0.4]{ext.png}
\end{figure}
\begin{figure}[h]
    \caption{\textbf{Restrained Eating}}
    \centering
    \includegraphics[scale=0.4]{res.png}
\end{figure}

\section{Discussion}
\subsection{Reliability}
The reliability of DEBQ as a whole questionnaire has been confirmed.
Although part \textbf{External Eating} always present lowest reliability among the three subscales\cite{nolan2010emotional,wu2017validation},
the result is still surprising. One possible reason for the extreme outcome is the removal of three less-critical questions.
Despite poor importance in the section, these three questions may still make non-negligible contributions to the internal consistency.
Further research should look into the problem, as a good balance between effectiveness and reliability should be achieved for design improvement.

\subsection{Relation Discovery}
\subsubsection{DEBQ scores}
To our surprise, gender differences are significant for DEBQ subscales. In all three sections, females score higher than males on average.
The difference is particularly prominent on \textbf{Emotional Eating} and \textbf{Restrained Eating},
while slight difference on \textbf{External Eating} is observed. Higher scores for the female is consistent with that from
research on Taiwanese parents\cite{wang2018psychometric}, although two samples didn't share similar age patterns.
Prior study on samples of the French elderly didn't find gender difference on the scores\cite{bailly2012dutch}.
This result challenged the reliability of DEBQ, as no gender difference should be observed to ensure consistent criteria.
We hypothesize that age probably could affect DEBQ scales systematically, as different interpretations or other differences could be influential.
This also means that in practice criteria applied to samples should be adjusted according to both age and gender. 
\subsubsection{DEBQ to Consumption}
As for consumption, there are several significant relations with DEBQ subscales. First, for snacks and late supper frequency,
\textbf{Emotional Eating} is a significant predictor, both for males and females. This is consistent with prior study that high
impulsivity contributes to high fatty food intake\cite{limbers2015executive}. Second, contrary to our hypothesis, none of the three
subscales present correlation with healthy food intake, measured by vegetable/fruit frequency in the past 4 days. The result implied statistically consistent vegetable and fruit
intake among Chinese student samples. We then propose that this outcome is affected by food structure differences. Then, strong positive relation between 
\textbf{External Eating} for females and high-fat food consumption is observed. However, given the context of this research, 
the possible explanation is not obvious. Finally, we found positive relation between high-sugar drink intake and \textbf{Emotional Eating, External Eating}
scores for females. Given soft drink as a kind of common unhealthy intake, this finding can also be interpreted as affected by low impulsivity control in females. \\
\smallskip

The overall results show mediocre effectiveness. Given DEBQ as a long-tested but solid eating questionnaire, this may come from the ineffectiveness
of SDSCA-type questions we designed. The four-day range could inhibit effectiveness and bring up randomness of the answers, while a larger range could bring up biases.
For self-reported questionnaires, randomness and self-report biases can both be influential, thus requiring balance between effectiveness and reliability for future test design.
Besides differences in eating structure and habits, the effectiveness of DEBQ itself is still questionable, as little structure has been found in male samples.
Also, unexpected high scores on \textbf{Restrained Eating} and \textbf{External Eating} for females are noticeable. These results challenged the reliability of DEBQ
as a general detection tool for unhealthy eating behaviors, since different understandings on the questions by college students, particularly female students, may have greatly affected the results. 
The vague differences between 'bad' and 'good' eating patterns, implied by the questionnaire, probably magnified the biases of the examinees.
Additionally, the self-report questionnaires present unavoidable drawbacks, as actual physical conditions of the tested samples are not clear. Detailed and behavior-based
research should be adapted for more in-depth study.

\bibliography{file} % 引用文件
\bibliographystyle{aaai} % 引用格式

\section{Acknowledgement}
This research is supported by my teammates and their selfless help. Gratefulness also goes to the supervisor Jing Chen, from whom
I learned valuable skills to finish this research. 

\end{document}