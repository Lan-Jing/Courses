%%%%%%%%%%%%%%%%%%%%%%%%%%%%%%%%%%%%%%%%%
% Jacobs Landscape Poster
% LaTeX Template
% Version 1.0 (29/03/13)
%
% Created by:
% Computational Physics and Biophysics Group, Jacobs University
% https://teamwork.jacobs-university.de:8443/confluence/display/CoPandBiG/LaTeX+Poster
% 
% Further modified by:
% Nathaniel Johnston (nathaniel@njohnston.ca)
%
% Modified further still by:
% Abraham Nunes (nunes <at> dal <dot> ca)
%
% License:
% CC BY-NC-SA 3.0 (http://creativecommons.org/licenses/by-nc-sa/3.0/)
%
%%%%%%%%%%%%%%%%%%%%%%%%%%%%%%%%%%%%%%%%%

%----------------------------------------------------------------------------------------
%	PACKAGES AND OTHER DOCUMENT CONFIGURATIONS
%----------------------------------------------------------------------------------------

\documentclass[final]{beamer}

\usepackage[scale=1.24]{beamerposter} % Use the beamerposter package for laying out the poster

\usetheme{confposter} % Use the confposter theme supplied with this template

\setbeamercolor{block title}{fg=black,bg=white} % Colors of the block titles
\setbeamercolor{block body}{fg=black,bg=white} % Colors of the body of blocks
\setbeamercolor{block alerted title}{fg=white,bg=black} % Colors of the highlighted block titles
\setbeamercolor{block alerted body}{fg=black,bg=white} % Colors of the body of highlighted blocks
% Many more colors are available for use in beamerthemeconfposter.sty

%-----------------------------------------------------------
% Define the column widths and overall poster size
% To set effective sepwid, onecolwid and twocolwid values, first choose how many columns you want and how much separation you want between columns
% In this template, the separation width chosen is 0.024 of the paper width and a 4-column layout
% onecolwid should therefore be (1-(# of columns+1)*sepwid)/# of columns e.g. (1-(4+1)*0.024)/4 = 0.22
% Set twocolwid to be (2*onecolwid)+sepwid = 0.464
% Set threecolwid to be (3*onecolwid)+2*sepwid = 0.708

\newlength{\sepwid}
\newlength{\onecolwid}
\newlength{\twocolwid}
\newlength{\threecolwid}
\setlength{\paperwidth}{48in} % A0 width: 46.8in
\setlength{\paperheight}{36in} % A0 height: 33.1in
\setlength{\sepwid}{0.024\paperwidth} % Separation width (white space) between columns
\setlength{\onecolwid}{0.22\paperwidth} % Width of one column
\setlength{\twocolwid}{0.464\paperwidth} % Width of two columns
\setlength{\threecolwid}{0.708\paperwidth} % Width of three columns
\setlength{\topmargin}{-0.5in} % Reduce the top margin size
%-----------------------------------------------------------

\usepackage{graphicx} % Required for including images

\usepackage{booktabs} % Top and bottom rules for tables

%----------------------------------------------------------------------------------------
%	TITLE SECTION 
%----------------------------------------------------------------------------------------

\title{Relations between Food and Drink Consumption and Eating Behavior Assessment DEBQ -- a Study on Chinese College Students} % Poster title
\author{Jing Lan} % Author(s)
\institute{lanj25@mail2.sysu.edu.cn \\ Sun Yat-sen University} % Institution(s)

%----------------------------------------------------------------------------------------

\begin{document}

\addtobeamertemplate{block end}{}{\vspace*{2ex}} % White space under blocks
\addtobeamertemplate{block alerted end}{}{\vspace*{2ex}} % White space under highlighted (alert) blocks

\setlength{\belowcaptionskip}{2ex} % White space under figures
\setlength\belowdisplayshortskip{2ex} % White space under equations

\begin{frame}[t] % The whole poster is enclosed in one beamer frame

\begin{columns}[t] % The whole poster consists of three major columns, the second of which is split into two columns twice - the [t] option aligns each column's content to the top

\begin{column}{\sepwid}\end{column} % Empty spacer column

\begin{column}{\onecolwid} % The first column

%----------------------------------------------------------------------------------------
%	OBJECTIVES
%----------------------------------------------------------------------------------------

\setbeamercolor{block alerted title}{fg=black,bg=white} % Change the alert block title colors
\setbeamercolor{block alerted body}{fg=black,bg=white} % Change the alert block body colors

\begin{alertblock}{Objectives}

\begin{enumerate}
	\item to replicate reliability of DEBQ on a sample set of Chinese college students.
	\item to discover relations between three DEBQ subscales, gender, and food/drink consumption, assessed by SDSCA-type scales.
	\item as a minor objective, positive relation between restrained eating and unhealthy eating patterns 
		  is expected. Gender differences will also be tested. 
\end{enumerate}

\end{alertblock}

%----------------------------------------------------------------------------------------
%	INTRODUCTION
%----------------------------------------------------------------------------------------

\begin{block}{Introduction}

Study began with: 
\begin{enumerate}
	\item characterizing distinct kinds of eating disorders
	\item inventing accurate diagnostic criteria
	\item effective measurement of abnormal symptoms
\end{enumerate}
The Eating Attitude Test(EAT) was the first self-reported questionnaire for evaluation \\
\indent Research further: reliability challenged. 
\begin{enumerate}
	\item clinical criteria for anorexia changed greatly
	\item more types separated: bulimia(overeating)
\end{enumerate}
Abnormal patterns <-- non-clinical samples \\
More questionnaires invented, including DEBQ. \\
Usually higher scores --> worse eating patterns \\
\bigskip
Questionnaires <-- theories on formation \\
Psychosomatic theory --> EMAQ, DEBQ: emotion \\
DEBQ --> high theoretical flexibility and reliability \\
\bigskip
Questions under-researched.
\begin{enumerate}
	\item wasn't translated, validated on Chinese env
	\item question not well: \textbf{baker}, habits, env...
	\item self-report biases --> low effectiveness
	\item posi relation: restrained --> unhealthy intake
\end{enumerate} 

\end{block}

%------------------------------------------------

%----------------------------------------------------------------------------------------

\end{column} % End of the first column

\begin{column}{\sepwid}\end{column} % Empty spacer column

\begin{column}{\twocolwid} % Begin a column which is two columns wide (column 2)

\begin{columns}[t,totalwidth=\twocolwid] % Split up the two columns wide column

\begin{column}{\onecolwid}\vspace{-.6in} % The first column within column 2 (column 2.1)

%----------------------------------------------------------------------------------------
%	MATERIALS
%----------------------------------------------------------------------------------------

\begin{block}{Participants}

Participants of this study were 52 college students from Sun Yat-sen University.
Demographic information is presented in Table 1. All 31 males and 21 females
were young Asian adults, with average BMI \(20.8~(\pm 2.67)\) for male samples and \(19.0~(\pm 1.82)\) for females.
Demographic analysis shows typicality of samples used for the research.
Few samples were discarded in demographic analysis as participants responded with abnormal values due to privacy concerns.
\renewcommand\tablename{\textbf{Table}}
\begin{table}[htbp]
    \centering
    \begin{tabular}{lcr} 
        \toprule
        Characteristics & average (\textit{N}) & precentage(range) \\
        \midrule
        Female (\textit{N}=21) &  & 0.404 \\ 
        Age (\textit{N}=20) & \(19.25~(\pm 0.44)\) & (19-20)\\ 
        BMI (\textit{N}=20) & \(19.1~(\pm 1.82)\) & (16.7-23.7) \\
        ~~~Underweight & (9) & 0.45 \\
        ~~~Normal Weight & (11) & 0.55 \\
        ~~~Overweight & (0) & 0 \\
        ~~~Obese & (0) & 0 \\
        \midrule
        Male (\textit{N}=31) &  & 0.596 \\ 
        Age (\textit{N}=28) & \(19.35~(\pm 0.56)\) & (19-20)\\ 
        BMI (\textit{N}=30) & \(20.8~(\pm 2.67)\) & (14.9-24.1)\\ 
        ~~~Underweight & (3) & 0.1 \\
        ~~~Normal Weight & (24) & 0.8 \\
        ~~~Overweight & (3) & 0.1 \\
        ~~~Obese & (0) & 0 \\
        \bottomrule
    \end{tabular}
\end{table}

\end{block}

%----------------------------------------------------------------------------------------

\end{column} % End of column 2.1

\begin{column}{\onecolwid}\vspace{-.6in} % The second column within column 2 (column 2.2)

%----------------------------------------------------------------------------------------
%	METHODS
%----------------------------------------------------------------------------------------

\begin{block}{Methods and Procedure}

Demo: age, gender, stature and weight\textbf{(--> BMI)} \\
\bigskip
Food/Drink Consumption: SDSCA-type questions.
\begin{table}[htbp]
    \centering
    \begin{tabular}{l}
        \toprule
        Questions (range:0-4) \\
        \midrule
        How many of the last four days did you: \\
        1. have snacks or late supper besides the normal meals? \\
        2. eat four or more servings of vegetables and fruits? \\
        3. eat any high-fat food or full-fat dairy product? \\
        4. drink high-sugar soft drink? \\
        \bottomrule
    \end{tabular}
\end{table}
\bigskip
DEBQ: 
\begin{enumerate}
	\item Emotional: 9 questions (4 removed <-- diffused)
	\item External: 7 questions (3 removed <-- less-critical)
	\item Restrained: 9 questions 
\end{enumerate}
\bigskip
Web-base questionnaire, questions randomly placed. \\
Demo --> consumption --> DEBQ \\
\bigskip
\textbf{SPSS} --> Reliability \\
\textbf{MATLAB} --> Demo analysis \\
\textbf{sklearn} --> Relation discovery \\
\begin{table}[htbp]
    \caption{\textbf{Reliability measured by \(\alpha\)}}
    \centering
    \begin{tabular}{lr}
        \toprule
        part & reliability \\
        \midrule
        DEBQ & 0.921 \\
        ~~Emotional Eating & 0.964 \\
        ~~Restrained Eating & 0.910 \\
        ~~External Eating & 0.587 \\
        \bottomrule
    \end{tabular}
\end{table}
Low on \textbf{External Eating}: removed 3 questions
\\ \textbf{balance} effectiveness <--> importance 

\end{block}

%----------------------------------------------------------------------------------------

\end{column} % End of column 2.2

\end{columns} % End of the split of column 2 - any content after this will now take up 2 columns width

%----------------------------------------------------------------------------------------
%	IMPORTANT RESULT
%----------------------------------------------------------------------------------------

\setbeamercolor{block alerted title}{fg=white,bg=dalgrey} % Change the alert block title colors
\setbeamercolor{block alerted body}{fg=black,bg=white} % Change the alert block body colors

\begin{alertblock}{Important Result}

\begin{enumerate}
	\item DEBQ reliability: high overall, low on \textbf{External Eating}
	\item gender dif: high DEBQ scores for girls
	\item \textbf{Emotional Eating} --posi--> snacks and late supper
	\item no relation for vegetable/fruit intake
	\item female \textbf{External Eating} --posi--> high-fat intake
	\item female \textbf{External, Emotional Eating} --posi--> high-sugar drink
\end{enumerate}

\end{alertblock} 

%----------------------------------------------------------------------------------------

\begin{columns}[t,totalwidth=\twocolwid] % Split up the two columns wide column again

\begin{column}{\onecolwid} % The first column within column 2 (column 2.1)

%----------------------------------------------------------------------------------------
%	MATHEMATICAL SECTION
%----------------------------------------------------------------------------------------

\begin{block}{}

\end{block}

%----------------------------------------------------------------------------------------

\end{column} % End of column 2.1

\begin{column}{\onecolwid} % The second column within column 2 (column 2.2)

%----------------------------------------------------------------------------------------
%	RESULTS
%----------------------------------------------------------------------------------------

\begin{block}{}

\end{block}

%----------------------------------------------------------------------------------------

\end{column} % End of column 2.2

\end{columns} % End of the split of column 2

\end{column} % End of the second column

\begin{column}{\sepwid}\end{column} % Empty spacer column

\begin{column}{\onecolwid} % The third column

%----------------------------------------------------------------------------------------
%	CONCLUSION
%----------------------------------------------------------------------------------------

\begin{block}{Results and Discussion}
\begin{figure}[h]
    \caption{\textbf{Gender differences}}
    \centering
    \includegraphics[scale=1]{res1.png}
\end{figure}
Reliability: gender diff? \\
age and gender --affect-> understanding \&\& results
\begin{figure}[h]
    \caption{\textbf{Emotional --> snacks and late supper}}
    \centering
    \includegraphics[scale=1]{emo1.png}
\end{figure}
\begin{figure}[h]
    \caption{\textbf{Emo, External(female) --> high-sugar drink}}
    \centering
    \includegraphics[scale=1]{ext1.png}
    \includegraphics[scale=1]{emo3.png}
\end{figure}
high impulsivity --> more unhealthy intake
\begin{figure}[h]
    \caption{\textbf{External(female) --> high-fat intake}}
    \centering
    \includegraphics[scale=1]{ext2.png}
\end{figure}
Low effectiveness: \\ 
\textbf{SDSCA} questions <-- randomness and biases \\
\textbf{DEBQ:}
\begin{enumerate}
    \item different eating env/culture/habits
    \item vague distinction and misunderstandings: \\ \textbf{bad} vs. \textbf{good} --> \textbf{restrained}
    \item innate drawback: no behavioral/physical test 
\end{enumerate}

\end{block}

%----------------------------------------------------------------------------------------
%	ADDITIONAL INFORMATION
%----------------------------------------------------------------------------------------



%----------------------------------------------------------------------------------------
%	REFERENCES
%----------------------------------------------------------------------------------------

%----------------------------------------------------------------------------------------
%	ACKNOWLEDGEMENTS
%----------------------------------------------------------------------------------------

\setbeamercolor{block title}{fg=black,bg=white} % Change the block title color

\begin{block}{Acknowledgements}

\small{\rmfamily{This research is supported by my teammates and their selfless help. Gratefulness also goes to the supervisor Jing Chen, from whom
I learned valuable skills to finish this research. }} \\

\end{block}

%----------------------------------------------------------------------------------------
%	CONTACT INFORMATION
%----------------------------------------------------------------------------------------

\setbeamercolor{block alerted title}{fg=white,bg=dalgrey} % Change the alert block title colors
\setbeamercolor{block alerted body}{fg=black,bg=white} % Change the alert block body colors

%\begin{alertblock}{Contact Information}
%\end{alertblock}

%\begin{center}
%\includegraphics[width=0.8\linewidth]{dal_fullmark_blak.jpg}
%\end{center}
%----------------------------------------------------------------------------------------

\end{column} % End of the third column

\end{columns} % End of all the columns in the poster

\end{frame} % End of the enclosing frame

\end{document}
